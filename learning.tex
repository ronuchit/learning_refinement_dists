\section{Learning Refinement Distributions}
In this section, we present our primary contribution: a method
for training good sampling distributions using reinforcement learning.
We first describe how to formulate our task and motion planning system
in the reinforcement learning framework. Then, we describe the training
process. Finally, we cover the details of the sampling distributions
and how to perform gradient updates for them.
\subsection{Formulation}
We define the state $s \in S$ to be the symbolic plan, its current refinement
(instantiation of values for all plan parameters), and the environment. Then, we define an action $a \in A$ as
a two-step process: 1) a discrete-space selection of which plan parameter $p$ to
resample, and 2) a continuous-space sampling of a new value for $p$. Under
these definitions, the transition function is deterministic. Also, we observe that
randomized refinement explicitly defines a fixed policy for
step 1, because it uses motion planning failure and precondition violation information
to guide a decision about which parameter to resample. Thus, our system focuses
on training a policy for step 2, by defining a sampling distribution (based on the
state) for each plan parameter.

Our reward function $R(a, s')$ provides constant rewards based on parameter resampling.
Because we check for IK feasibility when resampling, we provide -1 reward every
time we sample an IK-infeasible pose, and +3 reward for an IK-feasible pose.
Similarly, we give -3 reward upon motion planning failure or action precondition violation,
and +5 reward upon a successful call to the motion planner for an action. We do not
provide any bonus reward for being in a state where the entire plan refinement is valid.
This choice of $R$ implies that our training optimizes the sampling distributions for
motion planning feasibility, rather than explicitly for refining plans. However, the two are
closely correlated -- randomized refinement succeeds in fewer iterations when the sampling
distributions produce more motion planning feasible values.
\subsection{Training Process}
Our training system is defined by the following:
\begin{tightlist}
\item[$\Prob$:] A distribution over planning problems, encompassing both
the task planning problem and the environment.
\item[$\theta_{p}$:] The learned feature weight vector associated with the high
level plan parameter $p$. We learn one
weight vector per parameter \emph{type} (e.g. ``left gripper grasp pose'' or ``right
gripper putdown pose''), so that the learned distributions are
not tied to any single high level plan. These weight vectors encode the policy
we are training for step 2 of actions $a \in A$, as described above.
\item[$R$:] The reward function, as described above.
\item[$f(s, p, x), s \in S$:] A function mapping the current state $s$, symbolic
parameter $p$, and sampled value $x$ for $p$ to a feature vector. Currently, our feature vector includes
only geometric aspects of the sample and environment, such as distance between the sample and object being grasped.
\item[$N$:] The number of planning problems on which to train.
\item[$L$:] The number of samples for one planning problem.
\item[$\epsilon$:] The number of samples comprising an episode. We perform
a weight vector update after each episode.
\end{tightlist}

The training is a natural extension of randomized
refinement and progresses as follows. $N$ times, sample from $\Prob$ to obtain
a complete planning problem $\Pi$. For each $\Pi$, run the randomized refinement
algorithm to attempt to find a valid plan refinement, allowing the \texttt{resample}
routine to be called $L$ times. As the system takes on-policy actions $a \in A$ (based on
current $\theta$ values) for variable resampling, collected
rewards are appended to a global reward list. If a valid, motion planning
feasible refinement is found, we forcibly resample one of
the high level variables at random (potentially making the refinement infeasible again)
and continue. Every $\epsilon$ calls to
\texttt{resample}, we perform a gradient update on the weight vectors $\theta$ using the
collected rewards and samples since the previous update. This process
evinces the strength of randomized refinement for our system: only the variable
instantiations responsible for failures are resampled, so the learning is naturally
focused toward improving sampling distributions which are not yet well-shaped.

\subsection{Distribution and Weight Updates}
We now describe the particular distribution we use to sample
and how to perform batch gradient updates for the weight vector $\theta$ of
this distribution. Zucker et al.~\cite{workspacebias} use features of a discretization of the workspace to train
a configuration space sampler for motion planning, and we adapt their training
method to the case of a continuous distribution. Thus, we follow their definition of
the distribution for a parameter $p$ with sample value $x$, in state $s \in S$: $$q(s, p, x) \propto exp(\theta_{p}^{T} f(s, p, x))$$
(The probability distribution must be appropriately normalized so
that it integrates to 1.) A useful facet of this distribution is
that when $\theta_{p}$ is the zero vector, $q$ is uniform over the
sample space, providing a good initialization for the weight
vectors. We sample $x$ from $q$ using the Metropolis algorithm,
an MCMC technique for sampling from a probability distribution using
a symmetric proposal distribution. At each iteration, the sampled value
is based only on the previous one, so the samples form a Markov chain.

The distribution $q$ is exactly the policy we are training for step 2 of actions in our
reinforcement learning formulation: it describes how to sample a continuous-space value $x$
for a parameter $p$, given current state $s$. We have, thus, employed linear function
approximation of the optimal policy, since we learn a linear combination of the features.

Zucker et al.~\cite{workspacebias} define the expected reward of an episode $\xi$:
$$\eta(\theta_{p}) = E_{q}[R(\xi)]$$ and provide an approximation for its gradient:
$$\nabla \eta(\theta_{p}) \approx \frac{R(\xi)}{\epsilon} \sum_{i=0}^{\epsilon}(f(s, p, x_{i}) - E_{q,s}[f])$$
where $R(\xi)$ is the sum over all rewards obtained throughout $\xi$ and
$E_{q,s}[f]$ is the expected feature vector under $q$, in state $s$. Because our distribution
is continuous, we cannot easily calculate $E_{q,s}[f]$, and so we approximate it by averaging
together the feature vector of several samples from $q$. The weight vector update is then:
$$\theta_{p} \leftarrow \theta_{p} + \alpha \nabla \eta(\theta_{p})$$
for appropriate step size $\alpha$.
