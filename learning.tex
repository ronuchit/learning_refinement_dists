\section{Learning Refinement Distributions}
\subsection{Formulation and Training Process}
In this section, we present our primary contribution: a method
for training good sampling distributions using reinforcement learning.
Formally, our training system is defined by the following:
\begin{tightlist}
\item[$\Prob$:] A distribution over planning problems, encompassing both
the task planning problem and the environment.
\item[$\theta_{a}, a \in \A$:] The learned feature weight vector associated with the high
level action $a$, which defines the sampling distribution for the
parameters of $a$. In our formulation, there is one
weight vector per action \emph{type} (e.g. ``left gripper grasp,'' ``right
gripper grasp,'' or ``base motion''), so that the learned distributions are
not tied to any particular high level plan.
\item[$\R$:] The reward function.
\item[$f(a, s_{0}, ..., s_{n}), a \in \A$:] A function mapping the sampled
parameter values $s_{0}, ..., s_{n}$ of action $a$ to a feature vector representation.
\item[$N$:] The number of planning problems on which to train.
\item[$L$:] The number of samples for one planning problem.
\item[$\epsilon$:] The number of samples comprising an episode. We perform
a weight vector update after each episode.
\end{tightlist}

Our system attempts to learn the feature weights $\theta$ for each action type; these weights
encode a policy that maps the state (comprised of the planning problem $\Pi \in \Prob$
and the symbolic plan) to a set of actions governing plan refinement.

The training is a natural extension of randomized
refinement and progresses as follows. $N$ times, sample from $\Prob$ to obtain
a complete planning problem $\Pi$. For each $\Pi$, run the randomized refinement
algorithm to attempt to find a valid plan refinement, allowing the \texttt{resample}
routine to be called $L$ times. As the refinement process runs, any collected
reward is appended to a global reward list. If a valid, motion planning
feasible refinement is found, we collect appropriate rewards, then resample one of
the high level variables at random (potentially making the refinement infeasible)
and continue. Every $\epsilon$ calls to
\texttt{resample}, we perform a gradient update on the weight vectors using the
collected rewards and samples since the previous update. This process
evinces the strength of randomized refinement for our system: only the variable
instantiations responsible for failures are resampled, so the learning is naturally
focused toward improving sampling distributions which are not yet well-shaped.

\subsection{Distribution and Weight Updates}
We now describe the particular distribution we use to sample
and how to perform batch gradient updates for the weight vector of
this distribution. Zucker et al.~\cite{workspacebias} use features of a discretization of the workspace to train
a configuration space sampler for motion planning, and we adapt their
method to the case of a continuous distribution. Accordingly, we define the sampling distribution for
action $a$ as $$q(\theta_{a}, s) \propto exp(\theta_{a}^{T} f(a, s))$$
(The probability distribution must be appropriately normalized so
that it integrates to 1.) A useful facet of this distribution is
that when $\theta_{a}$ is the zero vector, $q$ is uniform over the
sample space, providing a good initialization for the weight
vectors. We sample from $q$ using the Metropolis algorithm,
an MCMC technique for sampling from a probability distribution using
a symmetric proposal distribution. (At each iteration, the sampled value
is based only on the previous one, so the samples form a Markov Chain.)

Zucker et al.~\cite{workspacebias} define the expected reward of an episode $e$:
$$\eta(\theta_{a}) = E_{q}[\R(e)]$$ and provide an approximation for its gradient:
$$\nabla \eta(\theta_{a}) \approx \frac{\R(e)}{\epsilon} \sum_{i=0}^{\epsilon}(f(a, s_{i}) - E_{q}[f])$$
where $\R(e)$ is the sum over all rewards obtained throughout the episode, and
$E_{q}[f]$ is the expected feature vector under $q$. Because our distribution
is continuous, we cannot easily calculate $E_{q}[f]$, and so we approximate it by averaging
together the feature vector for several samples from $q$. The weight vector update is then:
$$\theta_{a} \leftarrow \theta_{a} + \alpha \nabla \eta(\theta_{a})$$
for appropriate step size $\alpha$.

\subsection{Reward and Feature Functions}
Our reward function $\R$ provides small constant rewards based on the ongoing
plan refinement. Because we check for IK feasibility of samples during the initialization
step and when resampling, we provide a small negative reward every
time we sample an IK-infeasible pose, and a small positive reward whenever an IK check succeeds.
Similarly, we add negative reward upon motion planning failure or action precondition violation,
and positive reward upon a successful call to the motion planner. Finally, we assign
a large reward for finding a complete, valid refinement of the high level plan.
This choice of reward function implies that our training optimizes the sampling distributions for
motion planning feasibility, rather than explicitly for refining plans. However, the two are
closely correlated -- plan refinement takes fewer iterations when variable
instantiations are likely to be motion planning feasible. Our feature function $f$ considers only geometric
aspects of the sample, such as distance between the sample and object being grasped, and distance from
the sample to nearby obstructions.
