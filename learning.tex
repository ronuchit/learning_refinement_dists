\section{Learning Refinement Distributions}
In this section, we present an RL approach that learns a policy for plan refinement.

\subsection{Formulation as Reinforcement Learning Problem}
We formulate plan refinement as an MDP as follows:
\begin{tightlist}
\item A state $s \in \St$ is a tuple $(P, r_{cur}, E, n)$, consisting of the
symbolic plan, its current instantiation of values for symbolic references,
the geometric encoding of the environment, and a counter for
the number of calls to the sampler.
\item An action $a \in \A$ is a pair $(p, x)$, where $p$ is the discrete symbolic
reference to resample and $x$ is the continuous value assigned to $p$ in the new refinement.
\item The transition function $T(s, a, s')$ is split up into 3 cases. In all cases, $n$ increases by 1. $L$ refers to
the number of samples for one planning problem.
  \begin{tightlist}
  \item Case 1: $n > L$. We sample a new state from $\Prob$ and reset $n$ to 0.
  \item Case 2: the proposed value $x$ is IK infeasible. The state remains the same.
  \item Case 3: Otherwise, the value of $p$ is set to $x$ and the motion planner is called.
  \end{tightlist}
\item The reward function $R(s, a, s')$ provides rewards based on a measure of closeness to a valid plan refinement.
\item $\Prob$ is a distribution over planning problems, encompassing both the task planning problem
and the environment.
\end{tightlist}

We restrict our attention to training policies that suggest $x$ for actions in $\A$.
We note that randomized refinement provides a fixed policy for selecting $p$.

Our reward function $R$ explicitly encourages successful plan refinement, providing positive reward linearly
interpolated between 0 and 20 based on the fraction of high-level actions whose preconditions are
satisfied. Additionally, we give $-1$ reward every time we sample an IK infeasible pose,
to minimize how long the system spends resampling plan variables until obtaining IK feasible samples.

\subsection{Training Process}
We learn a policy for this MDP by adapting the method of Zucker et al.~\cite{workspacebias}, which
uses a linear combination of features to define a distribution over poses. In our setting, we learn a weight
vector $\theta_{p}$ for each reference \emph{type}, comprised of a pose type and possibly a gripper
(e.g., ``left gripper grasp pose,'' ``right gripper putdown pose,'' ``base pose'').
This decouples the learned distributions from any single high-level plan and allows generalization across problems.

We develop a feature function $f(s, p, x)$ that maps the current
state $s \in \St$, symbolic reference $p$, and sampled value $x$ for $p$ to a
feature vector; $f$ defines a policy class for the MDP. Additionally, we define
$N$ as the number of planning problems on which to train, and
$\epsilon$ as the number of samples comprising a training episode.

The training is a natural extension of randomized
refinement and progresses as follows. $N$ times, sample from $\Prob$ to obtain
a complete planning problem $\Pi$. For each $\Pi$, run the randomized refinement
algorithm to attempt to find a valid plan refinement, allowing the \textsc{Resample}
routine to be called $L$ times before termination. Select actions according to the $\theta_{p}$
and collect rewards according to $R$. After every $\epsilon$ calls to
\textsc{Resample}, perform a gradient update on the weights.

\subsection{Distribution and Gradient Updates}
We adopt the sampling distribution used in Zucker et al.~\cite{workspacebias}
for a symbolic reference $p$ with sample value $x$, in state $s \in \St$:
$$q(s, p, x) \propto exp(\theta_{p}^{T} f(s, p, x)).$$
The authors define the expected reward of an episode $\xi$:
$$\eta(\theta_{p}) = \mathbb{E}_{q}[R(\xi)]$$ and provide an approximation for its gradient:
$$\nabla \eta(\theta_{p}) \approx \frac{R(\xi)}{\epsilon} \sum_{i=1}^{\epsilon}(f(s, p, x_{i}) - \mathbb{E}_{q,s}[f]).$$
$R(\xi)$ is the sum over all rewards obtained throughout $\xi$, and
$\mathbb{E}_{q,s}[f]$ is the expected feature vector under $q$, in state $s \in \St$. The weight vector update is then:
$$\theta_{p} \leftarrow \theta_{p} + \alpha \nabla \eta(\theta_{p})$$
for appropriate step size $\alpha$.

We sample $x$ from $q$ using the Metropolis algorithm~\cite{chib1995understanding}.
Since our distributions are continuous, we cannot easily calculate $\mathbb{E}_{q}[f]$,
so we approximate it by averaging together the feature vectors for several samples from $q$.