\section{Randomized Refinement}
In order to move toward refining high level plans using continuous sampling distributions,
we present one contribution of our work: a novel algorithm called randomized refinement, which resamples
plan variables based on which values are leading to motion planning
failures or violations of action preconditions. We thus leverage information
about action precondition fluents explicitly, which backtracking refinement
does not take into account. In the next section, we will show how this new refinement strategy
can be easily augmented with a reinforcement learning system to train good
distributions for sampling values for plan variables. Algorithm 1 shows pseudocode
for randomized refinement.

The procedure takes two arguments, the high level plan object and a maximum
iteration count. In Line 2, all variables in the high level plan are initialized by sampling
from the corresponding distributions. For efficiency, we only initialize the symbolic pose
references (such as object grasping poses) with IK-feasible values, so as to avoid unnecessary calls to the
motion planner. We then call the \textsc{MotionPlan} subroutine in Line 4, which
iterates through the sequence of actions that comprise the high level plan.
For each action, a call to the motion planner is made (Line 22) using the instantiated values
for the parameters of that action, to attempt to find a trajectory
linking the sampled poses. If this succeeds, the preconditions of the action
are tested (Line 26). As part of this step, we verify that the trajectory returned by the
black box motion planner is collision-free, satisfying the precondition
that the trajectory is feasible in the environment. Thus, based
on the returned values of \textsc{MotionPlan}, we may distinguish
a motion planning failure from a precondition violation, for the current
refinement. We appropriately call the \texttt{resample} routine on the high level parameters
associated with the failure; this routine picks one of the parameters at random and
resamples it from its refinement distribution. Again for efficiency, we keep resampling until we
have an IK-feasible value. If we reach the iteration limit (Line 18),
we convert the most recent failure information into a symbolic representation, then raise it
to the classical planner, which will update its fluent state and provide a new
high level plan.

\begin{algorithm}
 \caption{Randomized refinement.} \label{alg-randref}
 \begin{algorithmic}[1]
  \Procedure{RandRef}{$HLP, N$}
  \State $init \leftarrow$ initRefinement($HLP$)
  \For {iter = 0, 1, ..., $N$}
  \State $failStep, failPred \leftarrow $\textsc{MotionPlan}($HLP$)
  \If {$failStep$ is null}
  \State \Return success
  \Else
  \If {$failPred$ is null}
  \State \# \emph{Motion planning failure.}
  \State $failAction \leftarrow HLP.ops[failStep]$
  \State resample($failAction.params$)
  \Else
  \State \# \emph{Action precondition violation.}
  \State resample($failPred.params$)
  \EndIf
  \EndIf
  \EndFor
  \State Raise failure to task planner, receive new plan.
  \EndProcedure

  \Procedure{MotionPlan}{$HLP$}
  \For {$op_{i}$ in $HLP.ops$}
  \State $res \leftarrow op_{i}$.motionPlan()
  \If {$res.failed$}
  \State \Return $i$, null
  \Else
  \State $failPred \leftarrow op_{i}$.checkPreconds()
  \If {$failPred$}
  \State \Return $i$, $failPred$
  \Else
  \State \# \emph{Found successful plan refinement.}
  \State \Return null, null
  \EndIf
  \EndIf
  \EndFor
  \EndProcedure
 \end{algorithmic}
\end{algorithm}

We emphasize the simplicity of randomized refinement over backtracking
refinement. Essentially, we maintain a set of instantiations over all high level
variables at all times, and we perform local resampling on the variable values based
on failure information. This technique provides several benefits over a backtracking
system. First, it remains unchanged in performance when moving from discretized to continuous
sampling distributions, while backtracking loses completeness and requires introduction
of heuristics dictating when to stop sampling for a particular variable and backtrack
to a previous one. Additionally, it allows the parameter instantiations for a particular action in
the plan to be influenced by those for a \emph{future} action. (For example, in a
pick-and-place task, it can make sense for the object's grasp pose to be sampled
conditionally on the current instantiation of the putdown pose, even though the putdown
appears after the grasp in the plan sequence.) Randomized refinement does, however,
require explicit storage of predicates in the interface layer, so that action preconditions
can be checked within the environment to determine which parameters get resampled next.