\section{Randomized Refinement}
Before we can apply RL to plan refinement, we must formulate it as an MDP.
We design our approach to imitate that of Zhang and Dietterich~\cite{JobShopSched}:
we initialize an infeasible refinement and use a randomized local search to propose
improvements. We maintain a value for each high-level symbolic reference.
At each iteration, a reference whose current value is leading to a motion planning failure or
action precondition violation is picked randomly and resampled.
Algorithm \ref{alg-randref} shows pseudocode for this randomized refinement.

The procedure takes two arguments, a high-level plan and a maximum
iteration count. In line 1, we initialize all references in the high-level plan by sampling
from their corresponding distributions, which can be hand-coded or learned, as described in Section VI. We continue sampling
until we find bindings for symbolic pose references that satisfy
inverse kinematics constraints (IK feasibility). Trajectories are
initialized as straight lines.

\begin{algorithm}[t]
\begin{small}
  \SetAlgoLined
  \DontPrintSemicolon
  \SetKwProg{myalg}{Algorithm}{}{}
  \SetKwProg{myproc}{Subroutine}{}{}
  \myalg{RandRef($HLP, N_{max}$)} {
  \nl $init \leftarrow$ \textsc{InitRefinement}($HLP$)\;
  \nl \For {iter = 0, 1, ..., $N_{max}$} {
  \nl $failStep, failPred \leftarrow $\textsc{MotionPlan}($HLP$)\;
  \nl \If {$failStep$ == NULL} {
  \tcc{\footnotesize Found valid plan refinement.}
  \nl return success }
  \nl \ElseIf {$failPred$ == NULL} {
  \tcc{\footnotesize Motion planning failure.}
  \nl $failAction \leftarrow HLP.ops[failStep]$\;
  \nl \textsc{Resample}($failAction.params$) }
  \nl \Else {
  \tcc{\footnotesize Action precondition violation.}
  \nl \textsc{Resample}($failPred.params$) } }
  \nl Raise failure to task planner, receive new plan. }
\end{small}
\label{alg-randref}
\vspace{-1.5 em}
\end{algorithm}

We call the \textsc{MotionPlan} subroutine in line 3, which attempts to
find a collision-free set of trajectories linking all pose instantiations.
To do so, it iterates through the sequence of actions that comprise the high-level plan.
For each, it first calls the motion planner to find a trajectory
linking the sampled poses. If this succeeds, it tests the action preconditions;
as part of this step, it checks that the trajectory is collision-free.

If \textsc{MotionPlan} is unsuccessful, it returns into $failStep$ the index of the action where failure occurred.
If this was due to a motion planning failure, $failPred$ is NULL. Otherwise,
the violated action precondition is stored into $failPred$.

We call the \textsc{Resample} routine on the symbolic parameters
associated with a failure; this routine picks one of these at random and
resamples its value. If we reach the iteration limit (l.11),
we convert the most recent failure information into a symbolic representation, then raise it
to the task planner, which will update its fluent state and provide a new
high-level plan.

Randomized refinement has two key properties. The first is a very explicit algorithm state.
We show in the next section that this allows for a straightforward MDP
formulation. This is also beneficial from an
engineering perspective, as the simplicity allows for easy debugging. The second is that
it allows the instantiations for a particular action in
the plan to be influenced by those for a \emph{future} action. For example, in a
pick-and-place task, it can make sense for the object's grasp pose to be sampled
conditionally on the current instantiation of the putdown pose, even though the putdown
appears after the grasp in the plan sequence. Thus, it is easy for plan refinement to
respond to long-term dependencies in continuous values of symbolic references.