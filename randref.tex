\subsection{A Randomized Algorithm for Plan Refinement}
In order to apply the complete planning algorithm described above, we
must provide definitions for each of the subroutines mentioned in
Alg.\,\ref{alg:complete}. {\sc sfrcra-14} uses a backtracking search
over a discrete set of instantiations to implement
\textsc{RefineNode}. We want to learn policies for refinement, so we
seek an algorithm that is more easily formulated as an {\sc mdp}. Our
method imitates that of Zhang and Dietterich~\cite{JobShopSched}: we
initialize an infeasible refinement and use a randomized local search
to propose improvements. Alg.\,\ref{alg:randref} shows pseudocode for
this refinement strategy, which implements \textsc{RefineNode} in
Alg.\,\ref{alg:complete}.

The algorithm takes as input a high-level plan and a maximum iteration
count.  In line 1, we initialize a (potentially invalid) refinement by
sampling from distributions associated with each symbolic
reference. We continue sampling until we find bindings that satisfy
inverse kinematics constraints ({\sc ik} feasibility). Trajectories
are initialized as straight lines.

\begin{algorithm}[t]
\begin{small}
  \SetAlgoLined
  \DontPrintSemicolon
  \SetKwProg{myalg}{Algorithm}{}{}
  \SetKwProg{myproc}{Subroutine}{}{}
  \myalg{RandRef($\pi, N_{max}$)} {
  \nl $\sigma \leftarrow$ \textsc{InitRefinement}($\pi$)\;
  \nl \For {iter = 0, 1, ..., $N_{max}$} {
  \nl $failStep, failPred \leftarrow $\textsc{GetMotionPlan}($\pi$)\;
  \nl \If {$failStep$ == NULL} {
  \tcc{\footnotesize Found valid plan refinement.}
  \nl return success }
  \nl \ElseIf {$failPred$ == NULL} {
  \tcc{\footnotesize Motion planning failure.}
  \nl $failAction \leftarrow \pi.ops[failStep]$\;
  \nl \textsc{Resample}($failAction.params$) }
  \nl \Else {
  \tcc{\footnotesize Action precondition violation.}
  \nl \textsc{Resample}($failPred.params$) } }}

\end{small}
\caption{Randomized local search for plan refinement.}
\label{alg:randref}
\end{algorithm}

The \textsc{MotionPlan} subroutine called in line 3 attempts to
find a collision-free set of trajectories linking all pose instantiations.
To do so, it iterates through the sequence of actions comprising the high-level plan.
For each, it first calls the motion planner to find a trajectory
linking the sampled poses. If this succeeds, it tests the action preconditions;
as part of this step, it checks that the trajectory is collision-free.

We then call the \textsc{Resample} routine on the symbolic parameters
associated with the infeasibility; this routine picks one at random and
resamples its value. \textsc{InitRefinement} and \textsc{Resample} together define
\textsc{NDGetInstantiation} for our implementation, while \textsc{GetError} iterates
through the steps of the plan, checks precondition and trajectory feasibility, and returns
a failed action index and associated predicate.

Randomized refinement has two key properties. The first is a very
explicit algorithm state.  We show in the next section that this
allows for a straightforward {\sc mdp} formulation. The second is that
it allows the instantiations for a particular action in the plan to be
influenced by those for a \emph{future} action. For example, in a
pick-place domain, it can make sense for the object's grasping pose to
be sampled conditionally on the current instantiation of the putdown
pose, even though the putdown appears after the grasp in the plan
sequence. This allows plan refinement to account for long-term
dependencies in symbolic reference instantiation.
