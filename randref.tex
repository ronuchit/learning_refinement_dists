\section{Randomized Refinement}
In order to move toward refining high level plans using continuous sampling distributions,
we present one contribution of our work: randomized refinement, a local search
algorithm for plan refinement. It maintains at all times a value for each high level plan
variable; at each iteration, a variable whose current value is leading to a motion planning failure or
action precondition violation is picked randomly and resampled locally. This method of making local improvements draws
parallels with the job shop scheduling algorithm and is straightforward to formulate
in the reinforcement learning framework, as we show in the next section.
Algorithm \ref{alg-randref} shows pseudocode for randomized refinement.

\begin{algorithm}
 \caption{Randomized refinement.} \label{alg-randref}
 \begin{algorithmic}[1]
  \Procedure{RandRef}{$HLP, N$}
  \State $init \leftarrow$ initRefinement($HLP$)
  \For {iter = 0, 1, ..., $N$}
  \State $failStep, failPred \leftarrow $\textsc{MotionPlan}($HLP$)
  \If {$failStep$ is null}
  \State \Return success
  \Else
  \If {$failPred$ is null}
  \State \# \emph{Motion planning failure.}
  \State $failAction \leftarrow HLP.ops[failStep]$
  \State resample($failAction.params$)
  \Else
  \State \# \emph{Action precondition violation.}
  \State resample($failPred.params$)
  \EndIf
  \EndIf
  \EndFor
  \State Raise failure to task planner, receive new plan.
  \EndProcedure

  \Procedure{MotionPlan}{$HLP$}
  \For {$op_{i}$ in $HLP.ops$}
  \State $res \leftarrow op_{i}$.motionPlan()
  \If {$res.failed$}
  \State \Return $i$, null
  \Else
  \State $failPred \leftarrow op_{i}$.checkPreconds()
  \If {$failPred$}
  \State \Return $i$, $failPred$
  \Else
  \State \# \emph{Found successful plan refinement.}
  \State \Return null, null
  \EndIf
  \EndIf
  \EndFor
  \EndProcedure
 \end{algorithmic}
\end{algorithm}

The procedure takes two arguments, the high level plan object and a maximum
iteration count. In Line 2, all variables in the high level plan are initialized by sampling
from the corresponding distributions. For efficiency, we only initialize the symbolic pose
references (such as object grasping poses) with IK-feasible values, so as to avoid unnecessary calls to the
motion planner. We then call the \textsc{MotionPlan} subroutine in Line 4, which
iterates through the sequence of actions that comprise the high level plan.
For each action, a call to the motion planner is made (Line 22) using the instantiated values
for the parameters of that action, to attempt to find a trajectory
linking the sampled poses. If this succeeds, the preconditions of the action
are tested (Line 26). As part of this step, we verify that the trajectory returned by the
black box motion planner is collision-free, satisfying the precondition
that the trajectory is feasible in the environment.

Thus, based on the returned values of \textsc{MotionPlan}, we may distinguish
a motion planning failure from a precondition violation, for the current
refinement. We appropriately call the \texttt{resample} routine on the high level parameters
associated with the failure; this routine picks one of the parameters at random and
resamples it from its refinement distribution. Again for efficiency, we keep resampling until we
have an IK-feasible value. If we reach the iteration limit (Line 18),
we convert the most recent failure information into a symbolic representation, then raise it
to the classical planner, which will update its fluent state and provide a new
high level plan.

We emphasize the benefits of randomized refinement. First, it
lends itself to easy formulation as an MDP, as we show in the next section.
Additionally, it allows the parameter instantiations for a particular action in
the plan to be influenced by those for a \emph{future} action. For example, in a
pick-and-place task, it can make sense for the object's grasp pose to be sampled
conditionally on the current instantiation of the putdown pose, even though the putdown
appears after the grasp in the plan sequence. Thus, it is easy for plan refinement to
respond to long-term dependencies in continuous values of plan variables. Finally, its simplicity
allows for ease of debugging.