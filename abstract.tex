\begin{abstract}
In mobile manipulation planning, it is not uncommon for tasks to require thousands of
individual motions. Planning complexity is exponential in the length
of the plan, rendering direct motion planning intractable for
many problems of interest. Recent work has focused on
\emph{task and motion planning} (TAMP) as a way to address this
challenge. TAMP methods integrate logical search with continuous
geometric reasoning in order to sequence several short-horizon motion plans that together
solve a long-horizon task. In this paper, we present machine learning techniques to improve the reliability
of planning in TAMP. The system we build on first plans symbolically,
and then performs \emph{plan refinement}, which determines feasible settings of values for
symbolic references to continuous parameters. Our methods are adaptable for other systems.
We present two improvements to current TAMP systems.
First, to account for continuous parameters, these systems typically
rely on hand-coded distributions to reduce the sample space. Such an approach lacks robustness and
requires substantial design effort. We thus use reinforcement learning (RL) to train
policies for plan refinement. Second, we develop a complete algorithm for TAMP
by introducing a \emph{plan refinement graph}, which allows
interleaving plan refinement with a search over \emph{which} plan to try refining,
given options that address different infeasibilities. We then use machine learning
to train heuristic functions that guide this search.
Our contributions are as follows: 1) we present a complete algorithm for TAMP;
2) we present a randomized local search algorithm for plan refinement
that is easily formulated as a Markov decision process (MDP); 3) we give an
RL algorithm that learns a policy for this MDP;
4) we present a method that trains heuristics for intelligently searching a plan refinement graph;
and 5) we perform experiments to evaluate the performance of our system in a variety of simulated
domains. We show significant improvements in success rate over the system we build on.
\end{abstract}
